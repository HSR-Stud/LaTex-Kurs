%BuG-Fix
%Package pdf Error: Driver file ................ not found
%If you have a luatex driver fail uncomment these lines
%\RequirePackage{luatex85}
%\def\pgfsysdriver{pgfsys-pdftex.def}

% Genereller Header
\documentclass[11pt,twoside,a4paper,fleqn]{article}
% Dateiencoding
\usepackage[utf8]{inputenc}
\usepackage[T1]{fontenc}	%ä,ü...
% Seitenränder
\usepackage[left=1cm,right=1cm,top=1cm,bottom=0.5cm,includeheadfoot]{geometry}
\setlength{\headsep}{12pt} 
% Sprachpaket 
\usepackage[english, ngerman]{babel} % Silbentrennung und Rechtschreibung Englisch und Deutsch

%%%%%%%%%%%%%%%%%%%%%%%
%% Wichtige Packages %%
%%%%%%%%%%%%%%%%%%%%%%%
\usepackage{amsmath}                % Allgemeine Matheumgebungen									
%\usepackage{amssymb}                % Fonts: msam,msbm, eufm & Mathesymbole, Mengen (lädt automatisch amsfonts)									
\usepackage{array}                  % \newcolumntype, \firsthline, ,\lasthline, m{width}, b{width}									
\usepackage{caption}                % Bildunterschriften									
\usepackage{enumitem}               % basic environments: enumerate, itemize, description									
\usepackage{fancybox}               % \fbox: \shad­ow­box, \dou­ble­box, \oval­box, \Oval­box									
\usepackage{fancyhdr}               % Seiten schöner gestalten, insbesondere Kopf- und Fußzeile									
\usepackage{floatflt}               % Textumflossene Abbildungen \begin{floatingfigure}[r]{Breite} : r rechts, l links, p links auf geraden Seiten und rechts auf ungeraden Seiten								
\usepackage{graphicx}               % \includegraphics[keyvals]{imagefile}, [draft]graphicx zeigt nur Namen und Rahmen an, [final] hebt diese option auf => Bild wird angezeigt    									
\usepackage{hyperref}               % Erstellt Verweise innerhalb und nach außerhalb eines PDF Dokumentes.									
\usepackage{lastpage}               % Bspw. : Page 1 of 3 => \thepage\ of \pageref{LastPage}									
\usepackage{listings}               % Erlaubt es Programmcode in der gewünschten Sprache zu hinterlegen (C++, Matlab,..). Definition der Sprache mit \lstset{language=name}..									
\usepackage{longtable}              % Longtable erlaubt es Tabellen zu erstellen die bei der nächsten Seite weiterlaufen. (Bricht automatisch um)									
%\usepackage{mathabx}                % Mathesymbole									
%\usepackage{mathrsfs}               % \mathscr (Benötigt für Fourierreihen-Symbol)									
%\usepackage{mathtools}              % Extension package to amsmath									
\usepackage{multicol}               % multicols-Umgebung \begin{multicols}{3} erzeugt Abschnitt mit 3 Spalten									
\usepackage{multirow}               % Tabelle: ermöglicht es Felder mehrerer Zeilen in einem zusammenzufassen									
\usepackage{pdflscape}              % adds PDF support to the environment 'landscape'									
\usepackage{pxfonts}                % Symbole, griechisches Alphabet, Integrale...									
\usepackage{rotating}               % sideways, turn{degree}, rotate{degree}, sidewaysfigure, sidewaystable Umgebung									
\usepackage{subcaption}             % Bildunterschriften für Subfigures									
\usepackage{tabularx}               % tabularx-Umgebung: Hat feste Gesamtbreite, \begin{tabularx}{\textwidth}{c c c c c} X: Spalte mit variabler Breite, l, c, r, p{breite}, m{breite}									
%\usepackage{textcomp}               % text symbols: baht, bullet, copyright, musical-note, onequarter, section, yen									
%\usepackage{tikz}                   % Tikz Umgebung zur Grafikerzeugung									
\usepackage{titlesec}               % Überschriften zu Textabstände
\usepackage{trfsigns}               % Transformationszeichen \laplace, \Laplace..									
\usepackage{trsym}                  % Weitere Laplace Zeichen erlaubt auch vertikale Transformationszeichen									
\usepackage{verbatim}               % verbatim, verbatim*, comment Umgebung									
\usepackage{wrapfig}                % Textumflossene Bilder und Tabellen, \begin{wrapfigure}[Zeilen]{Position}[Ueberhang]{Breite}									
\usepackage{xcolor}                 % \pagecolor{color}, \textcolor{color}{text}, \colorbox{color}{text}, \fcolorbox{border-color}{fill-color}{text}									
\usepackage{titlesec}
% Zum Bilder einfach in Tabellen einfügen (valign=t)
\usepackage[export]{adjustbox}

%%%%%%%%%%%%%%%%%%%%
% Generelle Makros %
%%%%%%%%%%%%%%%%%%%%
\newcommand{\skript}[1]{$_{\textcolor{red}{\mbox{\small{Skript S.#1}}}}$}
\newcommand{\verweis}[2]{\small{(siehe auch \ref{#1}, #2 (S. \pageref{#1}))}}
\newcommand{\verweiskurz}[1]{(\small{siehe \ref{#1}\normalsize)}}
\newcommand{\subsubadd}[1]{\textcolor{black}{\mbox{#1}}}
\newcommand{\formelbuch}[1]{$_{\textcolor{red}{\mbox{\small{S#1}}}}$}

\newcommand{\kuchling}[1]{$_{\textcolor{red}{\mbox{\small{Kuchling #1}}}}$}
\newcommand{\stoecker}[1]{$_{\textcolor{grey}{\mbox{\small{Stöcker #1}}}}$}
\newcommand{\sachs}[1]{$_{\textcolor{blue}{\mbox{\small{Sachs S. #1}}}}$}
\newcommand{\hartl}[1]{$_{\textcolor{green}{\mbox{\small{Hartl S. #1}}}}$}

\newcommand{\schaum}[1]{\tiny Schaum S. #1}

\newcommand{\skriptsection}[2]{\section{#1 {\tiny Skript S. #2}}}
\newcommand{\skriptsubsection}[2]{\subsection{#1 {\tiny Skript S. #2}}}
\newcommand{\skriptsubsubsection}[2]{\subsubsection{#1 {\tiny Skript S. #2}}}

\newcommand{\matlab}[1]{\footnotesize{(Matlab: \texttt{#1})}\normalsize{}}

\newcommand\tabbild[2][]{%
	\raisebox{0pt}[\dimexpr\totalheight+\dp\strutbox\relax][\dp\strutbox]{%
		\includegraphics[#1]{#2}%
	}%
}

\newcolumntype{P}[1]{>{\raggedright\arraybackslash}p{#1}} %Tabelle linksausgerichtet
\newcolumntype{L}[1]{>{\raggedleft\arraybackslash}p{#1}} %Tabelle rechtsausgerichtet
\newcolumntype{C}[1]{>{\centering\arraybackslash}p{#1}}



%%%%%%%%%%
% Farben %
%%%%%%%%%%
\definecolor{black}{rgb}{0,0,0}
\definecolor{red}{rgb}{1,0,0}
\definecolor{white}{rgb}{1,1,1}
\definecolor{grey}{rgb}{0.8,0.8,0.8}
\definecolor{green}{rgb}{0,.8,0.05}
\definecolor{brown}{rgb}{0.603,0,0}
\definecolor{mymauve}{rgb}{0.58,0,0.82}


%%%%%%%%%%%%%%%%%%%%%%%%%%%%
% Mathematische Operatoren %
%%%%%%%%%%%%%%%%%%%%%%%%%%%%
\DeclareMathOperator{\sinc}{sinc}
\DeclareMathOperator{\sgn}{sgn}
\DeclareMathOperator{\Real}{Re}
\DeclareMathOperator{\Imag}{Im}
%\DeclareMathOperator{\e}{e}
\DeclareMathOperator{\cov}{cov}
\DeclareMathOperator{\PolyGrad}{PolyGrad}

%Grösse Integral anpassen
\def\Int{\mbox{\Large$\displaystyle\int$\normalsize}}
\def\OInt{\mbox{\Large$\displaystyle\oint$\normalsize}}

%Makro für 'd' von Integral- und Differentialgleichungen 
\newcommand*{\diff}{\mathop{}\!\mathrm{d}}

%%%%%%%%%%%%%%%%%%%%%%%%%%%
% Fouriertransformationen %
%%%%%%%%%%%%%%%%%%%%%%%%%%%

% Fouriertransformationen
\unitlength1cm
\newcommand{\FT}
{
	\begin{picture}(1,0.5)
	\put(0.2,0.1){\circle{0.14}}\put(0.27,0.1){\line(1,0){0.5}}\put(0.77,0.1){\circle*{0.14}}
	\end{picture}
}


\newcommand{\IFT}
{
	\begin{picture}(1,0.5)
	\put(0.2,0.1){\circle*{0.14}}\put(0.27,0.1){\line(1,0){0.45}}\put(0.77,0.1){\circle{0.14}}
	\end{picture}
}


%%%%%%%%%%%%%%%%%%%%%%%%%%%%
% Allgemeine Einstellungen %
%%%%%%%%%%%%%%%%%%%%%%%%%%%%

\setitemize{noitemsep,topsep=0pt,parsep=0pt,partopsep=0pt} %kompakte itemize
\setenumerate{noitemsep,topsep=0pt,parsep=0pt,partopsep=0pt} %kompakte enumerate

%Pdf Info
\hypersetup{pdfauthor={\authorname},pdftitle={\titleinfo},colorlinks=false}
\author{\authorname}
\title{\titleinfo}

% Abstände Text zu Übertiteln / Einzug
%\titlespacing{\section}{12pt}{1em}{0.5em}
%\titlespacing{\subsection}{12pt}{1em}{0.5em}
%\titlespacing{\subsubsection}{12pt}{1em}{0.5em}

%%%%%%%%%%%%%%%%%%%%%%%
% Kopf- und Fusszeile %
%%%%%%%%%%%%%%%%%%%%%%%
\pagestyle{fancy}
\fancyhf{}
%Linien oben und unten
\renewcommand{\headrulewidth}{0.5pt} 
\renewcommand{\footrulewidth}{0.5pt}

%Kopfzeile links bzw innen
\fancyhead[L]{\titleinfo}
%Kopfzeile mitte
\fancyhead[C]{}
%Kopfzeile rechts bzw. aussen
\fancyhead[R]{\rightmark}

%Fusszeile links bzw. innen
\fancyfoot[L]{\footnotesize{\organizationinfo}}
%Fusszeile mitte
\fancyfoot[C]{\footnotesize{Seite \thepage { }von \pageref{LastPage}}}
%Fusszeile rechts bzw. ausen
\fancyfoot[R]{\footnotesize{\authorname}}
% Einrücken verhindern versuchen
\setlength{\parindent}{0pt}

%%%%%%%%%%%%%%%%%%%%%%%%%%%%%%%%%%%%%%%
%% Makros & anderer Low-Level bastel %%
%%%%%%%%%%%%%%%%%%%%%%%%%%%%%%%%%%%%%%%
% Zeilenhöhe Tabellen:
\newcommand{\arraystretchOriginal}{1.5}
\renewcommand{\arraystretch}{\arraystretchOriginal}

\makeatletter
%% Makros für den Arraystretch (bei uns meist in Tabellen genutzt, welche Formeln enthalten)
% Default Value
\def\@ArrayStretchDefault{1} % Entspricht der Voreinstellung von Latex

% Setzt einen neuen Wert für den arraystretch
\newcommand{\setArrayStretch}[1]{\renewcommand{\arraystretch}{#1}}

% Setzt den arraystretch zurück auf den default wert
\newcommand{\resetArrayStretch}{\renewcommand{\arraystretch}{\@ArrayStretchDefault}}

% Makro zum setzten des Default arraystretch. Kann nur in der Präambel verwendet werden.
\newcommand{\setDefaultArrayStretch}[1]{%
    \AtBeginDocument{%
        \def\@ArrayStretchDefault{#1}
        \renewcommand{\arraystretch}{#1}
    }
}
\makeatother

%% Achtung Symbol \danger
\newcommand*{\TakeFourierOrnament}[1]{{%
        \fontencoding{U}\fontfamily{futs}\selectfont\char#1}}
\newcommand*{\danger}{\TakeFourierOrnament{66}}

%%%%%%%%%%%%%%%%%%%%%%%%%%%%%%%%%%%%%%%%%%%%%%%%%%%%%%%%%%%%%
%% Paedi's Header Extension
%%%%%%%%%%%%%%%%%%%%%%%%%%%%%%%%%%%%%%%%%%%%%%%%%%%%%%%%%%%%%
\lstset
{		
		language=[LaTeX]TeX,
    breaklines=true,
    showspaces = false,
%    showstringspacs = false,
    basicstyle =  \tt,
    keywordstyle=\color{blue},
   	stringstyle=\color{brown},
    identifierstyle=\color{black},
}

\newcommand{\befehl}[1]{\texttt{\color{blue}$\backslash$#1}}
\newcommand{\umgebung}[1]{\texttt{\color{blue}#1}}

\definecolor{lstgrey}{rgb}{0.65,0.65,0.65}
\definecolor{aufcol}{rgb}{0, 0.6, 0}

\usepackage{pdfpages}
\usepackage{mdframed}
\mdfdefinestyle{mdCode}{
linecolor=lstgrey,
linewidth=1.5pt,
skipabove=6pt,
skipbelow=6pt,
innertopmargin=0.5\topskip,
innerbottommargin=0.5\topskip,
innerleftmargin=\topskip,
innerrightmargin=\topskip,
%innerlinewidth=0pt,
backgroundcolor=lstgrey!20,
%roundcorner=10pt
}
\mdfdefinestyle{mdAufgabe}{
linecolor=aufcol,
linewidth=1.5pt,
skipabove=6pt,
skipbelow=6pt,
innertopmargin=0.5\topskip,
innerbottommargin=0.5\topskip,
innerleftmargin=\topskip,
innerrightmargin=\topskip,
%innerlinewidth=0pt,
backgroundcolor=aufcol!20,
%roundcorner=10pt
}

\newcommand{\latexinput}[1]
{\begin{mdframed}[style=mdCode]\lstinputlisting{#1}\end{mdframed}}
\newcommand{\latexinputB}[1]
{\begin{mdframed}[style=mdCode]\begin{minipage}[h]{0.6\textwidth}
\lstinputlisting{#1}
\end{minipage}\vline \begin{minipage}[h]{0.05\textwidth}\end{minipage}
\begin{minipage}[h]{0.35\textwidth}
\input{#1}
\end{minipage}\end{mdframed}}
\newcommand{\latexinputC}[1]
{\begin{mdframed}[style=mdCode]\begin{minipage}[h]{0.6\textwidth}
\lstinputlisting{#1}
\end{minipage}\vline\hspace{0.5cm} %\begin{minipage}[h]{0.05\textwidth}\end{minipage}
\begin{minipage}[h]{0.35\textwidth}
\input{#1}
\end{minipage}\end{mdframed}}
\newcounter{aufgaben}
\setcounter{aufgaben}{1}
%\newcommand{\aufgabe}[2]{$ $\newline\textbf{$\blacktriangleright{}$ Aufgabe \theaufgaben: #1}\newline #2\stepcounter{aufgaben}}
\newenvironment{aufgabe}[1][\unskip]{%
\textbf{$\blacktriangleright{}$ Aufgabe \theaufgaben: #1}\newline}{\stepcounter{aufgaben}}
\surroundwithmdframed[style=mdAufgabe]{aufgabe}
%%%%%%%%%%%%%%%%
% Code Layout %
%https://en.wikibooks.org/wiki/LaTeX/Source_Code_Listings
%%%%%%%%%%%%%%%

\definecolor{mygreen}{rgb}{0,0.6,0}
\definecolor{mygray}{rgb}{0.5,0.5,0.5}
\definecolor{mymauve}{rgb}{0.58,0,0.82}

\lstset{ %
    firstnumber=1,
    backgroundcolor=\color{white},   % choose the background color; you must add        \usepackage{color} or \usepackage{xcolor}
    basicstyle=\footnotesize\ttfamily, % the size of the fonts that are used for the code
    breakatwhitespace=false,         % sets if automatic breaks should only happen at whitespace
    breaklines=true,                 % sets automatic line breaking
    captionpos=b,                    % sets the caption-position to bottom
    commentstyle=\color{mygreen},    % comment style
    deletekeywords={...},            % if you want to delete keywords from the given language
    otherkeywords={...},             % if you want to add more keywords to the set
    escapeinside={\%*}{*\%},          % if you want to add LaTeX within your code
    extendedchars=true,              % lets you use non-ASCII characters; for 8-bits encodings only, does not work with UTF-8
    frame=single,	                 % adds a frame around the code
    keepspaces=true,                 % keeps spaces in text, useful for keeping indentation of code (possibly needs columns=flexible)
    keywordstyle=\color{blue},       % keyword style
    language=C++,                    % the language of the code   
    numbers=left,                    % where to put the line-numbers; possible values are (none, left, right)
    numbersep=5pt,                   % how far the line-numbers are from the code
    numberstyle=\tiny\color{mygray}, % the style that is used for the line-numbers
    rulecolor=\color{black},         % if not set, the frame-color may be changed on line-breaks within not-black text (e.g. comments (green here))
    showspaces=false,                % show spaces everywhere adding particular underscores; it overrides 'showstringspaces'
    showstringspaces=false,          % underline spaces within strings only
    showtabs=false,                  % show tabs within strings adding particular underscores
    stepnumber=2,                    % the step between two line-numbers. If it's 1, each line will be numbered
    stringstyle=\color{mymauve},     % string literal style
    tabsize=2,	                     % sets default tabsize to 2 spaces
    %title=\lstname                   % show the filename of files included with         \lstinputlisting; also try caption instead of title
}

\lstdefinestyle{customc++}{
    belowcaptionskip=1\baselineskip,
    %frame=L,
    xleftmargin=\parindent,
    language=C++,
    keywordstyle=\bfseries\color{blue},
    commentstyle=\itshape\color{mygreen},
    identifierstyle=\color{black},
    stringstyle=\color{gray},
}

\lstdefinestyle{cppunit}{
    belowcaptionskip=1\baselineskip,
    %frame=L,
    xleftmargin=\parindent,
    language=C++,
    keywordstyle=\bfseries\color{blue},
    keywordstyle=[2]\bf\color{black}, %not sure why \bf works, but it does
    commentstyle=\itshape\color{mygreen},
    identifierstyle=\color{black},
    stringstyle=\color{gray},
    keywords=[2]{  %Cpp Unit Keywords
        CPPUNIT_ASSERT,
        CPPUNIT_TEST,
        CPPUNIT_TEST_EXCEPTION,
        CPPUNIT_TEST_END,
        CPPUNIT_TEST_SUITE,
        CPPUNIT_TEST_SUITE_REGISTRATION,
        CPPUNIT_TEST_SUITE_END},
}

\lstdefinestyle{cppqt}{
    belowcaptionskip=1\baselineskip,
    %frame=L,
    xleftmargin=\parindent,
    language=C++,
    keywordstyle=\bfseries\color{blue},
    keywordstyle=[2]\bfseries\color{red},
    commentstyle=\itshape\color{mygreen},
    identifierstyle=\color{black},
    stringstyle=\color{gray},
    keywords=[2]{           % qt-Keywords
		Qt,
        SIGNAL,
        SLOT,
        QApplication,
        QDialog,
        QGridLayout,
        QPushButton,
        QLabel,
        QVBoxLayout,
        QHBoxLayout,
        QWidget,
        QGroupBox,
        QFont,
        QLineEdit,
        QRadioButton,
        QPen,
        QRect,
        QPaintEvent,
        QBrush,
        QPixmap,
        QPainter,
        QString,
        QPoint,
        update()},
}

\lstdefinestyle{cdoxy}{
    belowcaptionskip=1\baselineskip,
    %frame=L,
    xleftmargin=\parindent,
    language=C++,  
    keywordstyle=\bfseries\color{blue},
    commentstyle=\itshape\color{mygreen},
    identifierstyle=\color{black},
    stringstyle=\color{gray},
    otherkeywords={           % DoxygenKeywords
        ...,
        ....,
        @mainpage,
        @file,
        @author,
        @version,
        @date,
        @bug,
        @brief,
        @extended,
        @param,
        @return,
        @warning,
        @note,
        @see},
}

\lstdefinestyle{custommatlab}{
	belowcaptionskip=1\baselineskip,
	%frame=L,
	xleftmargin=\parindent,
	language=Matlab,
	basicstyle=\footnotesize\ttfamily,
	keywordstyle=\bfseries\color{blue},
	commentstyle=\itshape\color{mygreen},
	identifierstyle=\color{black},
	stringstyle=\color{mylilas},
}

%choose customstyle in DOC with \lstinputlisting[style=custom]{path}
\lstset{style=customc++}


\usepackage{lipsum}
\usepackage{tocvsec2}
\usepackage{calc}
\usepackage{ifthen}
%\usepackage[landscape]{geometry}
\ifthenelse{\lengthtest { \paperwidth = 11in}}
	{ \geometry{top=.5in,left=.5in,right=.5in,bottom=.5in} }
	{\ifthenelse{ \lengthtest{ \paperwidth = 297mm}}
		{\geometry{top=1cm,left=1cm,right=1cm,bottom=1cm} }
		{\geometry{top=1cm,left=1cm,right=1cm,bottom=1cm} }
	}
\newcommand{\texstud}{\TeX studio }
\usepackage{tikzsymbols}