\section{InLab}
%Würde ich schon vorbereiten, damit nicht unnötig Zeit verloren geht.
		%	\begin{aufgabe}[Neues Dokument erstellen]
		%		Erstelle ein neues Dokument \textit{\glqq Tutorial.tex\grqq{}} und speichere es im Unterordner \textit{\glqq sections\grqq.}\\(Später werden wir dieses Dokument ins Hauptdokument einfügen.)
		%	\end{aufgabe}
		%	
		%	\begin{aufgabe}[Dokument einbinden]
		%		Binde das Dokument \textit{\glqq Tutorial.tex\grqq{}} in das Hauptdokument ein. (Dokumentpfad ist relativ anzugeben)
		%	\end{aufgabe}
	
	\begin{aufgabe}[Kapitel einfügen]
		\begin{itemize}
		\item[a)] Füge im Dokument \textit{\glqq erste\_Schritte.tex\grqq{}} ein neues Kapitel ein (z.B. \textit{Aufgaben}).
		\item[b)] Schreibe einen kurzen Text
		\end{itemize}
	\end{aufgabe}

	
	\begin{aufgabe}[Aufzählung erstellen]
		Erstelle folgende Aufzählung:
		\begin{itemize}	
			\item Argument 1
			\item Argument 2
			\begin{enumerate}
				\item Subargument 1
				\item Subargument 2
			\end{enumerate}
			\item Argument 3
		\end{itemize}
	\end{aufgabe}
	
		
	\begin{aufgabe}[Tic Tac Toe]
		Versuche mittels dem \href{https://tablesgenerator.com}{Internet} ein Tic Tac Toe-Spiel zu erstellen.\\
		Bsp:
		\begin{tabular}{c|c|c}
			x & o & o\\ \hline
			x & o & x\\ \hline
			x & x & o\\
		\end{tabular}
	\end{aufgabe}

	\begin{aufgabe}[Bild einfügen]
	Füge das Bild \textit{\glqq HSRLogo.jpg\grqq} (im Unterordner images zu finden) ein.
	\end{aufgabe}

	\begin{aufgabe}[Physikalisches Gausssches Gesetz]
	\begin{enumerate}
		\item Lade \href{https://mathpix.com}{Mathpix} herunter
		\item Starte Mathpix
		\item Drücke Ctrl+Alt+m und kopiere diesen Ausdruck:
	\end{enumerate}

	\[ \text{div } \vec{D} = \vec{\nabla} \cdot \vec{D} = \rho \Leftrightarrow \oiint_{\partial V}\vec{D}\cdot d\vec{A} = \iiint_V \rho \cdot dV = Q(V) \]
	Experimentiere auch mit der \textit{inline}-Mathematikumgebung, um die unterschiedlichen Darstellungen zu sehen. 
	\end{aufgabe}
	

\section{PostLab}

	\begin{aufgabe}[Ausprobieren!]
	\LaTeX{} lässt sich am besten lernen, indem man ausprobiert und andere Files analysiert. Clone dazu weitere Formelsammlungen von {\color{blue}\href{https://github.com/HSR-Stud/VorlageZFLaTex}{GitHub}}, schau dir an wie diese erstellt wurden und passe sie nach deinem Geschmack an. Falls Fragen auftauchen, dann versuch dein Problem zu googeln.\vspace{6pt}\\
	\textbf{\large Viel Spass beim Lernen! %\tikzsymbolsuse{Smiley}[1]
	}
\end{aufgabe}